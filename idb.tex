\documentclass[xetex]{beamer}
\usepackage{fontspec}
\usepackage{xunicode}
\usepackage{xltxtra}
\usepackage[polish]{babel}
\usepackage{tabularx}
\usepackage{color}
\usepackage{alltt}
\usepackage{hyperref}
\usepackage{multirow}
\usepackage{relsize}
\usepackage{listings}
\usepackage{graphicx}
\usepackage{ulem}
\usepackage{tikz}
\usetikzlibrary{shapes, arrows, positioning, matrix, fit, backgrounds, calc}

\defaultfontfeatures{Scale=MatchLowercase}
\setsansfont{Ubuntu}
\setmonofont{DejaVu Sans Mono}

\definecolor{lightgray}{rgb}{.9,.9,.9}
\definecolor{darkgray}{rgb}{.4,.4,.4}
\definecolor{purple}{rgb}{0.65, 0.12, 0.82}

\lstdefinelanguage{JavaScript}{
  keywords={typeof, new, true, false, catch, function, return, null, catch, switch, var, if, in, while, do, else, case, break},
  keywordstyle=\color{blue}\bfseries,
  ndkeywords={class, export, boolean, throw, implements, import, this},
  ndkeywordstyle=\color{darkgray}\bfseries,
  identifierstyle=\color{black},
  sensitive=false,
  comment=[l]{//},
  morecomment=[s]{/*}{*/},
  commentstyle=\color{purple}\ttfamily,
  stringstyle=\color{red}\ttfamily,
  morestring=[b]',
  morestring=[b]"
}

\lstset{
   language=JavaScript,
   extendedchars=true,
   basicstyle=\footnotesize\ttfamily,
   showstringspaces=false,
   showspaces=false,
   numbers=left,
   numberstyle=\footnotesize,
   numbersep=9pt,
   tabsize=2,
   breaklines=true,
   showtabs=false,
   captionpos=b
}

\usetheme{default}
\setbeamertemplate{navigation symbols}{}

\title{Save your data}
\subtitle{Introduction to IndexedDB}
\author{
    Paweł Maciejewski\\
    @fragphace
}
\date{}

\begin{document}

    \tikzstyle{block} = [draw, line width=1pt, rectangle, rounded corners, inner sep = 3pt, align=center]
    \tikzstyle{dbBlock} = [block, draw=red, fill = red!50]
    \tikzstyle{storeBlock} = [block, draw=blue, fill = blue!50]
    \tikzstyle{dbArrow} = [->, shorten <=10pt, shorten >=10pt, line width = 0.5pt]

    \begin{frame}[plain] 
      \titlepage
      \vspace{-2cm} 
    \end{frame}    
    
    \begin{frame}{Agenda}    

        \begin{enumerate}
          \item Why Indexed Database?
          \item Structure
          \item Transactions
          \item Database creation
          \item Basic IndexedDB pattern
          \item There's (much) more
        \end{enumerate}     
        
    \end{frame}

    \begin{frame}[containsverbatim]{Why Indexed Database?}
      \textbf{Answer 1}\\
      There are only Web Storage and Indexed Database (Web SQL is deprecated).
      In many cases Local Storage is too simple.

      \vspace{20pt}

      \textbf{Answer 2}\\
      It is good for your programming skills to play with asynchronous API of Indexed Database.
    \end{frame}

    \begin{frame}{Structure}

      \begin{itemize}        
        \item IDB enables (more) advanced key-value data management,
        \item Many databases associated with one origin, many object stores associated with one database
        \item It's NoSQL: you must take care of mapping relations between data, there is no full text search...
      \end{itemize}

      \begin{center}
        \begin{tikzpicture}

            % NODES
            % Origin
            \node[block, fill=green!50, draw=green] (origin) { \includegraphics[width=32pt, height=32pt]{img/Website-Home-64.png} };
            \node[align=center, width = 10pt] [below = 5pt of origin] {
              \textbf{Origin}
            };

            % Databases
            \node (db1) at ($ (origin) + (90pt, 0) $) {\includegraphics [width=32pt, height=32pt] {img/Misc-Database-3-icon.png}};
            \node (db2) at ($ (db1) - (0, 40pt) $)  {\includegraphics [width=32pt, height=32pt] {img/Misc-Database-3-icon.png}};    
            \node[text width=100pt, align=center] at ($ (db2) - (0, 35pt) $) {
              \textbf{Databases}
            };

            % Object stores
            \node (store1) at ($ (db1) + (90pt, 0) $) {\includegraphics[width=32pt, height=32pt]{img/Briefcase-256.png}};
            \node (store2) [below = 15pt of store1] {\includegraphics[width=32pt, height=32pt]{img/Briefcase-256.png}};
            \node (store3) [below = 5pt of store2] {\includegraphics[width=32pt, height=32pt]{img/Briefcase-256.png}};            
            \node[text width=100pt, align=center] (stores) [below = 5pt of store3] {
              \textbf{Object stores}
            };

            % ARROWS 
            % Database arrows
            \path [dbArrow] (origin.east) edge [bend left] (db1.west);
            \path [dbArrow] (origin.east) edge [bend right] (db2.west);

            % Object stores arrows
            \path [dbArrow] (db1.east) edge [bend left] (store1.west);
            \path [dbArrow] (db2.east) edge [bend left] (store2.west);          
            \path [dbArrow, shorten <= 25pt] (db2.east) edge [bend right] (store3.west);   


            % BACKGROUND LAYER
            \begin{pgfonlayer}{background}
              \node [dbBlock, fit = (db1) (db2) ] { };  
              \node [storeBlock, fit = (store1)] { };  
              \node [storeBlock, fit = (store2) (store3)] { }; 
            \end{pgfonlayer}
            
        \end{tikzpicture}

      \end{center}
    \end{frame}

    \begin{frame}[containsverbatim]{Transactions}
      
      \begin{itemize}
        \item Every request must be performed within a transaction
        \item Transaction scope determines the object stores with witch transaction can interact
        \item There are three modes of transactions:
          \begin{itemize}
            \item \verb|READ_ONLY|
            \item \verb|READ_WRITE|
            \item \verb|VERSION_CHANGE|
          \end{itemize}
      \end{itemize}

      \begin{center}
        \begin{tikzpicture}
            \node (req1) {Request};
            \node [below of = req1] (req2) {Request};
            \node [below of = req2] (req3) {Request};
            \node [draw=gray, line width=1pt, rectangle, inner sep = 3pt, fill=gray!40,  right = of req2] (trans) {Transaction};

            % Transaction arrows            
            \path [dashed] (req1.east) edge (trans.west);
            \path [dashed] (req2.east) edge (trans.west);
            \path [dashed] (req3.east) edge (trans.west);    
            
            \begin{pgfonlayer}{background}
              \node [draw, rounded corners, inner sep = 5pt, draw, fit = (req1) (req2) (req3) (trans) ] { };  
            \end{pgfonlayer}  
        \end{tikzpicture}
      \end{center}


    \end{frame}

    \begin{frame}[containsverbatim]{Database creation}    
      \begin{lstlisting}
// Let's specify db version
var connectionReq = window.indexedDB.open('myDB', 3); 

connectionReq.onsuccess = function(event) {};
connectionReq.onerror = function(event) {};

connectionReq.onupgradeneeded = function(event) {
  var connection = event.target.result;  
  var storeNames = connection.objectStoreNames;

  if (storeNames.contains('workers')) {
    // Delete an object store
    connection.deleteObjectStore('workers');
  }

  if (!storeNames.contains('employees')) {
    // Create an object store
    connection.createObjectStore('employees', { keyPath: 'email'}); 
  }
};
      \end{lstlisting}
    \end{frame}

    \begin{frame}[containsverbatim]{Basic IndexedDB pattern}
      \begin{lstlisting}
var connectionReq = window.indexedDB.open('myDB', 3);

connectionReq.onsuccess = function(event) {
  var connection = event.target.result;
  var transaction = connection.transaction('employees', IDBTransaction.READ_WRITE);
  var employees = transaction.objectStore('employees');

  var setReq = employees.set({ 
    email: 'fragphace@gmail.com',
    name: 'Pawel Maciejewski'
  });

  var getReq = employees.get('fragphace@gmail.com');

  // ...
};
      \end{lstlisting}
    \end{frame}

    \begin{frame}[containsverbatim]{There's (much) more}
      \begin{itemize}
        \item Cursors, bounds, indexes, cross-window events
        \item Not ready for production, new specification implemented only in Firefox 12
        \item Prefixes for \verb|indexedDB|, \verb|IDBTransaction| and others        
      \end{itemize}
    \end{frame}
    
    \begin{frame}[plain]
      \begin{center}
        \Huge Thank you!
      \end{center}
    \end{frame}

            
\end{document}
